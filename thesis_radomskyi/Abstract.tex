\documentclass[thesis.tex]{subfiles}

\begin{document}
\chapter{Abstract}

The human brain is an essential, yet so unresearched organ. Its uniqueness and immense complexity brings limitations to the accuracy of the diagnosis obtained through standard medical approaches. This is where modern achievements in the field of computer science come into play. Data processing algorithms and pipelines, combined with modern methods of acquiring diffusion-weighted images allow to compute measures that are specific for the brain tissue, perform their parameterization and analysis. CoBundleMap is one of such dMRI processing pipelines. It performs joint parametrization of white matter bundles across a given population, utilizing manifold learning for extracting bundle-specific diffusivity measures. Along with the Fixel-based analysis, an established and robust technique, which is capable of identifying local differences and correlations in the structure of white matter, the aforementioned system plays a significant role as a basis for this work. The presented thesis investigates the ways of extending the CoBundleMap image processing pipeline with a set of metrics that could be computed in the presence of crossing fibers. Such metrics include apparent fiber density (AFD), fiber cross-section (FC) and a combined fiber density and cross-section (FDC) measure. Evaluation of the applicability of the achieved results is also presented in the work, providing insightful background for analysis and future work.

\textbf{Keywords:} Neuroinformatics, CoBundleMAP, Diffusion MRI, Fixel-based analysis

\thispagestyle{empty}
\end{document}
