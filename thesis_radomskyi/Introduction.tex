\documentclass[thesis.tex]{subfiles}

\begin{document}

\chapter{Introduction}

The main objective of this work is to investigate the possibility of extending CoBundleMap system
with new metrics such as apparent fiber density (AFD), fiber cross-section (FC) and a combined fiber density and cross-section (FDC).

\section{Motivation}

CoBundleMap is a system for performing classification, regression and hypothesis testing using diffusion MRI data. It is based on features derived from diffusion parameters and the anatomical structure of the brain tissue. Motivation of this work is to improve, or at least investigate the possibility of improving it by
extending the existing feature vector with new measures that describe thickness and integrity of white matter bundles.


\section{Objectives}

Objectives of this work:
\begin{itemize}
  \item Implement AFD, FC and FDC computation mechanisms as a part of a processing pipeline
  \item Extend the existing CoBundleMap pipeline with computed fixel-based metrics
  \item Evaluate the applicability of an extended feature vector compared to the existing one
\end{itemize}

\section{Description of the work}

In essence, the complete work could be divided into two parts - theoretical and practical.
Theoretical research focuses on general exploration of new metrics and, most importantly, on finding a way of implementing
required computations without significantly disrupting of the existing pipeline logic as well as an overview of the pipeline itself. The aim of the practical task is to build the described theoretical solution and evaluate the possibility of using new metrics for improving CoBundleMap.


%\label{ch:introduction} 

\end{document}
